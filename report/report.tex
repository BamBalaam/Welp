\documentclass[a4paper,10pt]{article}
\usepackage[french]{babel}
\usepackage[utf8]{inputenc}
\usepackage[left=2.5cm,top=2cm,right=2.5cm,nohead,nofoot]{geometry}
\usepackage{url}
\usepackage[T1]{fontenc}
\usepackage{float}
\usepackage{afterpage}
\usepackage{amsmath}
\usepackage{graphicx}
\usepackage{tabularx}
\usepackage{csquotes}
\usepackage{fullpage}
\usepackage{pdfpages}
\usepackage{listings}
\usepackage{color}

\usepackage{listings}
\usepackage{color}

\definecolor{mygreen}{rgb}{0,0.6,0}
\definecolor{mygray}{rgb}{0.5,0.5,0.5}
\definecolor{mymauve}{rgb}{0.58,0,0.82}

\lstset{ %
  backgroundcolor=\color{white},   % choose the background color; you must add \usepackage{color} or \usepackage{xcolor}
  basicstyle=\footnotesize,        % the size of the fonts that are used for the code
  breakatwhitespace=false,         % sets if automatic breaks should only happen at whitespace
  breaklines=true,                 % sets automatic line breaking
  captionpos=b,                    % sets the caption-position to bottom
  commentstyle=\color{mygreen},    % comment style
  deletekeywords={...},            % if you want to delete keywords from the given language
  escapeinside={\%*}{*)},          % if you want to add LaTeX within your code
  extendedchars=true,              % lets you use non-ASCII characters; for 8-bits encodings only, does not work with UTF-8
  frame=single,                    % adds a frame around the code
  keepspaces=true,                 % keeps spaces in text, useful for keeping indentation of code (possibly needs columns=flexible)
  keywordstyle=\color{blue},       % keyword style
  language=Octave,                 % the language of the code
  otherkeywords={*,...},           % if you want to add more keywords to the set
  numbers=left,                    % where to put the line-numbers; possible values are (none, left, right)
  numbersep=5pt,                   % how far the line-numbers are from the code
  numberstyle=\tiny\color{mygray}, % the style that is used for the line-numbers
  rulecolor=\color{black},         % if not set, the frame-color may be changed on line-breaks within not-black text (e.g. comments (green here))
  showspaces=false,                % show spaces everywhere adding particular underscores; it overrides 'showstringspaces'
  showstringspaces=false,          % underline spaces within strings only
  showtabs=false,                  % show tabs within strings adding particular underscores
  stepnumber=2,                    % the step between two line-numbers. If it's 1, each line will be numbered
  stringstyle=\color{mymauve},     % string literal style
  tabsize=2,                     % sets default tabsize to 2 spaces
  title=\lstname                   % show the filename of files included with \lstinputlisting; also try caption instead of title
}

\usepackage[pdftex,
            pdfauthor={A. Caccia, A. Madeira Cortes},
            pdftitle={Bases de données - Projet},
            pdfsubject={INFO-H-303 : Bases de données - Projet}]{hyperref}

\linespread{1.1}

\setlength{\parskip}{0.5em}

\begin{document}

\begin{titlepage}
    \begin{center}
        \textbf{\textsc{Université Libre de Bruxelles}}\\
        \textbf{\textsc{Faculté des Sciences}}\\
        \textbf{\textsc{Département d'Informatique}}

        \vfill{}
        \vfill{}

        \begin{center}
            {\Huge INFO-H-303 : Bases de données}
        \end{center}

        {\Huge \par}

        \begin{center}
            {\LARGE Projet : Annuaire d'établissements Horeca}
        \end{center}

        {\Huge \par}

        \begin{center}
            {\large A. Caccia, A. Madeira Cortes}
        \end{center}

        {\Huge \par}
        \vfill{}
        \vfill{}

        {\large\par}
        \vfill{}
        \vfill{}
        % \enlargethispage{3cm} % do not remove

        \textbf{Année académique 2015--2016}
    \end{center}
\end{titlepage}

\section{Introduction}

Le projet consiste en la création d'une plateforme permettant de lister, commenter et noter des établissements horeca. Le but pédagogique étant, bien sur, d'interfacer cette plateforme avec une base de données crée par nos soins, il était important d'utiliser un langage de programmation permettant d'utiliser des requêtes SQL écrites de la même façon que celles vue au cours.

Le language que nous avons choisi est, in fine, Ruby. Le framework web est Rails (plus connu sous le nom de Ruby on Rails). Nous avons opté de choisir une base de données PostgreSQL, qui est la plus intuitive à utiliser avec Rails (et dont les requêtes conservent le format vu au cours).

\includepdf[pages={-},angle=90,scale=0.75,pagecommand=\section{Diagramme entité-asssociation}]{./../erdiagrams/ERDia.pdf}

\section{Traduction relationnelle}

\noindent User(\underline{ID}, Email, Password, DateSignUp, IsAdmin)

\hspace{-0,5cm}Place(\underline{ID}, Name, Address, GPS, PhoneNum, Website)

\hspace{-0,5cm}Restaurant(\underline{PID}, PriceRange, Banquet, TakeOut, Delivery, Closed)

PID reference Place.ID

\hspace{-0,5cm}Cafe(\underline{PID}, Smoking, Snack)

PID reference Place.ID

\hspace{-0,5cm}Hotel(\underline{PID}, NumStars, NumRooms, PriceRangeDoubleRoom)

PID reference Place.ID

\hspace{-0,5cm}Comment(\underline{PID}, \underline{UID}, \underline{Date}, Score, Text, Date)

PID reference Place.ID

UID reference User.ID

\hspace{-0,5cm}Tag(\underline{PID}, \underline{UID}, Name)

PID reference Place.ID

UID reference User.ID

\hspace{-0,5cm}Creates(\underline{PID}, \underline{UID}, CreationDate)

PID reference Place.ID

UID reference User.ID

\section{Contraintes d'intégrité}

\begin{itemize}
  \item Pour pouvoir créer, modifier ou supprimer un "Place", le "User" doit être un administrateur (\emph{IsAdmin} == True).
  \item La \emph{Date} d'un "Comment" doit être strictement supérieure à la \emph{CreationDate} d'un "Place".
  \item La \emph{Date} d'un "Comment" doit être strictement supérieure à la \emph{DateSignUp} du "User" qui le poste.
  \item Il ne peut exister deux "Comment" d'un même "User" pour un même "Place" à la même \emph{Date}.
  \item La \emph{Note} d'un "Comment" doit être comprise entre 0 et 5 inclus.
  \item Les \emph{Name} des "Tag" doivent être uniques.
  \item Un "User" ne peut apposer plus d'une fois un "Tag" sur une même "Place".
  \item Tous les \emph{Email} des "User" doivent être uniques.
\end{itemize}

\newpage

\section{Script SQL de création de la BDD}

\lstinputlisting[language=SQL]{./../src/creation.sql}

\newpage

\section{Requêtes demandées}

\subsection{Requête 1}

\textbf{Tous les utilisateurs qui apprécient au moins 3 établissements que l’utilisateur "Brenda" apprécie.}

\subsection{Requête 2}

\textbf{Tous les établissements qu’apprécie au moins un utilisateur qui apprécie tous les établissements que "Brenda" apprécie.}

\subsection{Requête 3}

\textbf{Tous les établissements pour lesquels il y a au plus un commentaire.}

\subsection{Requête 4}

\textbf{La liste des administrateurs n’ayant pas commenté tous les établissements qu’ils ont crées.}

\subsection{Requête 5}

\textbf{La liste des établissements ayant au minimum trois commentaires, classée selon la moyenne des scores attribués.}

\subsection{Requête 6}

\textbf{La liste des labels étant appliqués à au moins 5 établissements, classée selon la moyenne des scores des établissements ayant ce label.}

\section{Instructions d'installation de l'application}

\section{Scénario de démonstration de l'application}

\section{Apports personnels}

\section{Explications des choix et hypothèses}

\end{document}
